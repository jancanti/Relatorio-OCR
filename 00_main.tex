% CONFIGURACAO DOCUMENTO
\documentclass[12pt, a4paper]{article}
\usepackage[top=2cm, bottom=1.5cm, left=2.5cm, right=2.5cm]{geometry}
\renewcommand{\baselinestretch}{1.5}
\setlength{\parindent}{1cm}
\usepackage{setspace}
\usepackage{indentfirst}
\usepackage[utf8]{inputenc}
\usepackage[T1]{fontenc}
\usepackage{amsfonts}
\usepackage[brazil]{babel}
\renewcommand{\familydefault}{\sfdefault}
\usepackage{helvet}

% CONFIGURACAO CODIGO
\usepackage{listings}

% CONFIGURACAO REFERENCIAS
\usepackage{natbib}

% CONFIGURACAO IMAGENS
\usepackage{graphicx}

% GERAR TEXTO
\usepackage{lipsum}

\title{APS 5º Semestre}
\author{Daniel Jançanti}
\date{2017}

\begin{document}
	\begin{center}
	\Huge Universidade Paulista\\
	\large Ciência da Computação
	
	\vfill
	
	\large Atividade Prática Supervisionada\\
	\textbf{\MakeUppercase{DESENVOLVIMENTO DE UM SISTEMA PARA}}\\
	\textbf{\MakeUppercase{RECONHECIMENTO DE ESCRITA A MÃO}}
	
	\bigskip
	\bigskip
	
	\normalsize{
		Aleksander Rocha - R.A.: C630IH-0\\
		Daniel Jançanti - R.A.: C630IG-2\\
		Ingrid Oliveira - R.A.: C51791-7\\
		Rafaela Aranas - R.A.: C29CII-0\\
	}
	
	\vfill
	
	Araraquara -- \the\year

\end{center}

\thispagestyle{empty}
	\include{02_indice}
	\section{Objetivo e motivação do trabalho}
	Este trabalho apresenta um estudo sobre o reconhecimento visual de caracteres através da utilização das redes neurais. São abordados os assuntos referentes ao processamento digital de imagens, aos sistemas de reconhecimento de caracteres, e às redes neurais.
	
	Em relação ao processamento digital de imagens são apresentados temas, abrangendo os assuntos referentes à aquisição de imagens, ao tratamento e reconhecimento de padrões.
	\section{Introdução}
	Segundo \cite{osorio1991estudo}, um sistema de reconhecimento de caracteres permite que o computador possa adquirir informações através da leitura de textos, ou seja, ao invés de entrar as informações pelo teclado, é possível implementar um sistema computadorizado, associado a um dispositivo ótico, para a aquisição automática de informações descritas sob a forma textual.
	
	Os sistemas de reconhecimento de caracteres possuem uma grande importância junto ao processamento de dados. Isto é, devido ao fato da linguagem escrita ser a forma mais usual do ser humano armazenar e transmitir informações.
	
	As \textbf{redes neurais} são um novo paradigma de desenvolvimento de sistemas, que tem se difundido muito na atualidade. As redes neurais tem se apresentado como uma solução muito adequada para sistemas de reconhecimento de padrões, como é o caso de um sistema de reconhecimento de caracteres, onde os padrões os para serem reconhecidos são os próprios caracteres. Esta característica, a ser demonstrada em maiores detalhes posteriormente, foi o que levou ao estudo e emprego das redes neurais junto a este trabalho.
	
	O \textbf{processamento de imagens} é uma área de estudos da computação que tem crescido muito nos últimos anos. Seu grande crescimento se deve principalmente ao desenvolvimento de equipamentos, cada vez mais sofisticados e baratos, utilizados para a captura e tratamento de imagens digitais. O processamento gráfico engloba as área de processamento de imagem e computação gráfica, onde o processamento de imagens está relacionado ao tratamento e reconhecimento de padrões em imagens, e a computação gráfica está ligada à síntese de imagens e à geração de descrições (dados não pictóricos) utilizadas na obtenção dessas imagens.
	
	O objeto de trabalho no processamento de imagens é a imagem digital, onde esta é definida como sendo uma matriz de M x N elementos (vetores com informações referentes aos pontos da imagem). Cada elemento da imagem digital é denominado de pixel, tendo associado a si uma informação referente à luminosidade e à cor. Esta informação pode ser um valor indicativo de intensidade luminosa. Esta apresentação através de uma matriz bidimensional de pontos é resultante da manipulação da imagem como sendo uma área de memória do computador. A cada pixel é associada uma ou mais posições de memória, as quais deve armazenar as diversas informações referentes a estes. Com isto pode-se armazenar e processar as informações referentes a uma imagem para posteriormente exibi-la.
	
	O \textbf{processo de digitalização} consiste em realizar a aquisição de uma cena, a qual é passada para o computador em um formato adequado para que este possa manupulá-la. As informações visuais são convertidas em sinais elétricos por sensores óticos, e esses sinais são quantificados em valores binários e armazenados na memória do computador. No processo de digitalização, os sinais são amostrados espacialmente e quantificados em amplitude, de forma a obter a imagem digital.
	
	No processo de digitalização, a imagem sofrerá uma amostragem e uma discretização da intensidade luminosa, que é denominada de quantização. No \textbf{processo de quantização}, uma imagem com tons contínuos é convertida em uma de tons discretos. Para o armazenamento e processamento por um computador, cada tonalidade (intensidade da luz refletida por cada ponto da imagem) é representada por um valor armazenado de forma binária. Cada ponto amostrado possuirá portanto um valor binário correspondente à intensidade luminosa da imagem naquele ponto. As imagens podem ser do tipo monocromáticas ou policromáticas. Nas imagens monocromáticas somente uma faixa de comprimentos de onda (uma cor) é analisada pelo sensor, determinando as intensidades de luminosidade para esta faixa. As intensidades de luminosidades descritas acima são denominadas de tonalidades ou níveis de intensidade de cor. Para as imagens policromáticas a digitalização é feita para diferentes faixas de comprimento de onda (diferentes cores).
	\section{Sistemas de reconhecimento de caracteres}
	
	Segundo \cite{silva2003reconhecimento}, os sistemas de reconhecimento ótico de caracteres são sistemas desenvolvidos para, de certa forma, reproduzir a capacidade humana de ler textos.

	\subsection{Origens e Evolução}
	
		\cite{man1986pattern} disse que as origens da tentativa de sumular a leitura humana datam de 1870, quando foi inventado por Carey, o scanner de retina. A partir da evolução dos dispositivos capazes de captar e traduzir imagens em sinais que podiam ser medidos, novos horizontes foram descobertos. Entretanto, a primeira tentativa bem sucedida de reconhecimento de caracteres foi realizada somente em 1990 pelo cientista russo Tyutin, que desenvolveu um sistema que visava o auxílio de deficientes visuais.
	
		Alguns dos motivos do desenvolvimento de um sistema de reconhecimento de caracteres são:
		
		\begin{itemize}
			\item Aquisição de dados numéricos comerciais;
			\item Compactação de dados;
			\item Leitura automática de formulários;
			\item Identificação de endereçamento postal;
			\item Reconhecimento de cheques bancários;
			\item Sistemas de aquisição de textos para a tradução automática;
			\item Interface homem x máquina mais natural e sem necessidade de recodificação dos dados (sujeitos a um menor número de erros, pois não é necessário redigitar os textos, introduzindo novos erros);
			\item Auxílio a deficientes visuais, na leitura de textos (tradução e impressão automática em braile)
		\end{itemize}

	\subsection{Tipos de Sistemas de Reconhecimento de Caracteres}
		
		Os sistemas de reconhecimento de caracteres podem ser desenvolvidos utilizando-se diferentes procedimentos tanto de aquisição de dados como de processamento de informações além de serem orientados para diferentes tipos de aplicações.
	
		Podem ser classificados primeiramente quanto ao tipo de mecanismo utilizado na aquisição dos textos a serem reconhecidos. Nesta categoria, encontram-se os sistemas magnéticos, sistemas mecânicos e sistemas óticos.
	
		Dentro dos sistemas OCR, são encontradas duas classes importantes, aqueles que são baseados no processamento sequencial das informações e os baseados no processamento paralelo. Existem três categorias princiapais: sistema de reconhecimento online, sistema de reconhecimento de caracteres isolados e sistema de reconhecimento de escrito cursiva.
	
	\begin{figure}[!htb]
		\centering
		\includegraphics[scale=0.4]{img/organograma-tipos.jpg}
		\caption{Organograma}
		\label{Organograma}
	\end{figure}
	
	\subsection{Descrição dos Sistemas OCR}
		
		Existem diferentes etapas de processamento das informações visando o reconhecimento de textos por um sistema OCR. Estas etapas se estendem desde a captura da imagem do texto até a obtenção final de uma identificação deste. Em um sistema de COR não se deve considerar apenas a parte referente às técnicas empregadas na resolução do problema de classificação, mas um item de extrema importância é a medicação e avaliação dos resultados obtidos por estes.
		
		\subsubsection{Etapas de Processamento}
			
			Em geral, os sistemas de OCR possuem implementadas as seguintes funções: aquisição da imagem do texto, tratamento da imagem, localização e separação dos caracteres, pré-processamento dos padrões, extração de atributos, reconhecimento/classificação dos padrões e pós-processamento.
		
		\begin{figure}[!htb]
			\centering
			\includegraphics[scale=0.5]{img/etapas-de-processamento.jpg}
			\caption{Etapas do processo de reconhecimento}
			\label{Etapas do processo de reconhecimento}
		\end{figure}
		
		Para a separação dos caracteres, o algoritmo tem que levar em consideração diversos fatores, entre eles: o espaçamento entre caracteres e tamanho do caractere (fixo ou variável), a utilização de uma grade auxiliar (no caso de formulários com espaçamento bem definidos), e o tipo de texto a ser reconhecido, onde podem haver caracteres que se tocam ou caracteres com descontinuidade (características muito importante para módulos de segmentação). Esta etapa de processamento é até hoje uma das etapas mais críticas, sendo um problema ainda não completamente solucionado. \cite{koerich2004reconhecimento}
		
		Após terem sido isolados os caracteres, e preferencialmente, com a realização de um ajuste de tamanho e posição pré-definidos, pode-se então realizar uma etapa de extração de atributos. Esta etapada é opcional e dependerá exclusivamente do tipo de técnica empregada no reconhecimento e classificação dos padrões.
		
		\subsubsection{Técnicas de Reconhecimento}
			
			Existem diferentes técnicas empregadas no reconhecimento de padrões, mas estas podem ser classificadas em três grandes categorias, técnicas baseadas em:
			
			\begin{description}
				\item \textbf{Atributos globais}: Nesta categoria, os atributos (features) são extraídos de cada ponto interior ao retângulo que circunscreve a matriz do caractere a qual também pode ser denominado de \textit{frame}. Os atributos empregados não refletem nenhuma propriedade local, geométrica ou topológica do próprio traçado. Algumas dessas técnicas foram desenvolvidas originalmente para reconhecer apenas caracteres impressos.
				\item \textbf{Distribuição de pontos}: É uma outra forma de reduzir a dimensionalidade dos conjuntos de atributos, onde estes são derivados a partir das distribuições estatísticas dos pontos. Diferentes tipos de distribuições tem sido utilizadas correspondendo a diferentes técnicas de reconhecimento.
				\item \textbf{Atributos geométricos e topológicos}: Esta técnica é baseada na extração de atributos que descrevem a geometria ou topologia de interesse no traçado do caractere. Estes atributos podem representar propriedades locais. Esta é de longe a mais popular técnica estudada pelos pesquisadores. 
			\end{description}
	
		\subsubsection{Avaliação do Reconheimento}
			
			Segundo \cite{erpen2004reconhecimento}, uma das formas de realizar a avaliação destes sistemas foi através da criação de bancos de dados padrões para análise de desempenho, contendo amostras variadas de pares a serem reconhecidos par um novo algoritmo submetido a avaliação de alguns destes bancos de amostras de pares de caracteres se tornaram muito difundidos e utilizados na avaliação de diferentes sistemas. Dois destes bancos de dados muito conhecidos são: o de Highleyman e o de Munson. Estes bancos de dados estão disponíveis junto ao IEEE e fornecem um conjunto de caracteres manuscritos em letra de impressora. Os dados de Munson possuem 12760 amostras de 46 tipos diferentes de caracteres, com uma resolução de 24 x 24 pixels, e os dados de Highleyman possuem 1800 amostras de letras e 500 de numerais, com um total de 36 tipos diferentes de caracteres, com uma resolução de 12 x 12 pixels. Esta resolução (12x12) infelizmente não é a mais adequada, uma vez que para a reconhecimento de caracteres manuscritos é aconselhada uma matriz com dimensões em torno de 20 x 25 pixels. O maior problema referente a tais bancos de dados 6 a divulgação de sua existência e as formas de acesso aos mesmos. Infelizmente, para a realização deste trabalho não foi possível a utilização destas bases de dados de pares de caracteres, devido primeiramente ao fato de serem orientados a sistemas de reconhecimento de caracteres manuscritos e, em segundo, porque não foi possível até o momento ter acesso a eles.
			
			Os principais parâmetros a serem tomados como referenda são os índices de avaliação de desempenho de sistemas de OCR. Estes índices consistem de três indicadores: taxa de caracteres reconhecidos corretamente, taxa de caracteres substituídos (classificações incorretas) e taxa de rejeição (caracteres não identificados).

	\subsection{Redes Neurais}
		
		As redes neurais foram desenvolvidas a partir de uma tentativa de criar um modelo que descrevesse a estrutura e o funcionamento dos neurônios do cérebro humano. Os modelos de redes neurais buscam definir novos computadores ou novos modelos de processamento de dados. Estes devem apresentar um comportamento baseado em modelos neurobiológicos ao invés de modelos baseados em "circuitos de silício" (portas lógicas, circuitos combinacionais, biestáveis, etc).
		
		\subsubsection{Características e Aplicações das Redes Neurais}
		
			Disse \cite{nunes2004seleccao}, as redes neurais possuem a característica de serem muito apropriadas ao reconhecimento de padrões. Como já foi visto, as redes neurais podem sofrer um aprendizado, modificando seu comportamento frente a um conjunto de estímulos de entrada (padrão de entrada). Portanto, a rede pode aprender a dar uma resposta específica Para um determinado conjunto de estímulos fornecidos. Isto será obtido através da alteração dos pesos de atuação das entradas. 124 Logo as redes neurais são muito adequadas para reconhecimento e classificação de pares, pais podem se adaptar para responder a um padrão específico. As redes neurais possuem também uma alta velocidade de processamento, devido ao seu maciço paralelismo interno, possibilitando o desenvolvimento de certos tipos de aplicações complexas que necessitam operar em tempo real. Devido às características inerentes às redes neurais, estas podem realizar alguns tipos de tarefas que não são executadas de uma forma satisfatória em sistemas computacionais tradicionais, mas que para a ser humano são tarefas triviais. Elas possuem a característica de se adequar perfeitamente às seguintes aplicações:
			
			\begin{itemize}
				\item Reconhecimento e síntese contínua da fala;
				\item Reconhecimento visual de padrões e padrões em geral 
				\item Reconhecimento e classificação de imagens como, por exemplo: textos, assinaturas, impressões digitais, objetos, etc;
				\item Processamento adaptativo de sinais e eliminação de rut dos;
				\item Aplicações onde as dados fornecidos são incompletos e os resultados produzidos são aproximados
			\end{itemize}
	\newpage
\section{Plano de desenvolvimento da aplicação}

	Implementamos e desenvolvemos essa aplicação durante nosso estudo com processamento de imagens e OCR (reconhecimento ótico de caracteres). Já está conseguindo reconhecer palavras completas com espaços e quebra de linhas. Assim conseguimos reconhecer qualquer letra ou palavras do nosso alfabeto, nosso projeto foi desenvolvido com o intuito de utilizar duas matérias muito importantes durante esse 6º módulo, que são Processamento de Imagem, e Sistema de Informação Inteligentes, dessas matérias retiramos conhecimento para o desenvolvimento dessa aplicação. 
	
	A ideia é fazer o sistema reconhecer letras e palavras do alfabeto escrita a mão, para que funcionasse perfeitamente treinamos uma imagem do nosso alfabeto escrito a mão. Para que a aplicação aprenda todas as letras, e reconheça as palavras que forem escritas a mão, toda vez que quisermos que o sistema entenda uma palavra diferente, precisamos treiná-la, para que o sistema aprenda e com isso reconheça a palavra na imagem. O processo para que a aplicação leia a palavra, começa quando recortamos as letras do alfabeto que o sistema já reconhece e formamos a palavra, importamos para o programa para treiná-la, após o treinamento, pedimos para o sistema reconhecer a palavra que está na imagem, ele reconhece a palavra corretamente. Qualquer letra ou palavra do alfabeto que for colocado no sistema para reconhecimento, será reconhecida, no log do projeto é visualizado letra por letra a porcentagem de aprendizado do programa, assim verificamos como se fosse um gráfico para o aprendizado de cada letra que o programa teve, podendo ter uma ideia da qualidade do sistema que foi desenvolvido, até o momento todas palavras e letras foram reconhecidas com êxito. 

\begin{figure}[!htb]
	\centering
	\includegraphics[scale=0.5]{img/01-reconhecimento-letras-basicas.jpg}
	\caption{Reconhecimento de letras básicas}
	\label{Reconhecimento}
\end{figure}
	\section{Projeto}
\lipsum

\begin{figure}[!htb]
	\centering
	\includegraphics[scale=0.4]{img/organograma-tipos.jpg}
	\caption{Organograma}
	\label{Organograma}
\end{figure}

	\section{Código fonte}
\lstset{language=Java}
{\tiny 
	{\Large CharacterImage.java}
	\lstinputlisting{code/CharacterImage.java}

	{\Large CharacterProcessing.java}
	\lstinputlisting{code/CharacterProcessing.java}

	{\Large Classification.java}
	\lstinputlisting{code/Classification.java}
	
	{\Large OCRProcessing.java}
	\lstinputlisting{code/OCRProcessing.java}

	{\Large OCRView.java}
	\lstinputlisting{code/OCRView.java}
	
	{\Large PixelsPositionsClassification.java}
	\lstinputlisting{code/PixelsPositionsClassification.java}

	{\Large Position.java}
	\lstinputlisting{code/Position.java}
}



	\include{09_programa}
	\include{10_bibliografia}
\end{document}

1. Capa: identificando o curso, o tema, a relação de alunos do grupo (nome/RA)
2. Índice
3. Objetivo e motivação do trabalho
4. Introdução
5. Fundamentos das principais técnicas biométricas (conceitos gerais)
6. Plano de desenvolvimento da aplicação (elementos e ferramentas utilizadas)
7. Projeto (estrutura e módulos que serão desenvolvidos) do programa
8. Relatório com as linhas de código do programa
9. Apresentação do programa em funcionamento em um computador, apresentando
todas as funcionalidades pedidas e extras.
10. Bibliografia
11. Ficha de Atividades Práticas Supervisionadas