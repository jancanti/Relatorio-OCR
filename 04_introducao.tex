\section{Introdução}
	Segundo \cite{osorio1991estudo}, um sistema de reconhecimento de caracteres permite que o computador possa adquirir informações através da leitura de textos, ou seja, ao invés de entrar as informações pelo teclado, é possível implementar um sistema computadorizado, associado a um dispositivo ótico, para a aquisição automática de informações descritas sob a forma textual.
	
	Os sistemas de reconhecimento de caracteres possuem uma grande importância junto ao processamento de dados. Isto é, devido ao fato da linguagem escrita ser a forma mais usual do ser humano armazenar e transmitir informações.
	
	As \textbf{redes neurais} são um novo paradigma de desenvolvimento de sistemas, que tem se difundido muito na atualidade. As redes neurais tem se apresentado como uma solução muito adequada para sistemas de reconhecimento de padrões, como é o caso de um sistema de reconhecimento de caracteres, onde os padrões os para serem reconhecidos são os próprios caracteres. Esta característica, a ser demonstrada em maiores detalhes posteriormente, foi o que levou ao estudo e emprego das redes neurais junto a este trabalho.
	
	O \textbf{processamento de imagens} é uma área de estudos da computação que tem crescido muito nos últimos anos. Seu grande crescimento se deve principalmente ao desenvolvimento de equipamentos, cada vez mais sofisticados e baratos, utilizados para a captura e tratamento de imagens digitais. O processamento gráfico engloba as área de processamento de imagem e computação gráfica, onde o processamento de imagens está relacionado ao tratamento e reconhecimento de padrões em imagens, e a computação gráfica está ligada à síntese de imagens e à geração de descrições (dados não pictóricos) utilizadas na obtenção dessas imagens.
	
	O objeto de trabalho no processamento de imagens é a imagem digital, onde esta é definida como sendo uma matriz de M x N elementos (vetores com informações referentes aos pontos da imagem). Cada elemento da imagem digital é denominado de pixel, tendo associado a si uma informação referente à luminosidade e à cor. Esta informação pode ser um valor indicativo de intensidade luminosa. Esta apresentação através de uma matriz bidimensional de pontos é resultante da manipulação da imagem como sendo uma área de memória do computador. A cada pixel é associada uma ou mais posições de memória, as quais deve armazenar as diversas informações referentes a estes. Com isto pode-se armazenar e processar as informações referentes a uma imagem para posteriormente exibi-la.
	
	O \textbf{processo de digitalização} consiste em realizar a aquisição de uma cena, a qual é passada para o computador em um formato adequado para que este possa manupulá-la. As informações visuais são convertidas em sinais elétricos por sensores óticos, e esses sinais são quantificados em valores binários e armazenados na memória do computador. No processo de digitalização, os sinais são amostrados espacialmente e quantificados em amplitude, de forma a obter a imagem digital.
	
	No processo de digitalização, a imagem sofrerá uma amostragem e uma discretização da intensidade luminosa, que é denominada de quantização. No \textbf{processo de quantização}, uma imagem com tons contínuos é convertida em uma de tons discretos. Para o armazenamento e processamento por um computador, cada tonalidade (intensidade da luz refletida por cada ponto da imagem) é representada por um valor armazenado de forma binária. Cada ponto amostrado possuirá portanto um valor binário correspondente à intensidade luminosa da imagem naquele ponto. As imagens podem ser do tipo monocromáticas ou policromáticas. Nas imagens monocromáticas somente uma faixa de comprimentos de onda (uma cor) é analisada pelo sensor, determinando as intensidades de luminosidade para esta faixa. As intensidades de luminosidades descritas acima são denominadas de tonalidades ou níveis de intensidade de cor. Para as imagens policromáticas a digitalização é feita para diferentes faixas de comprimento de onda (diferentes cores).